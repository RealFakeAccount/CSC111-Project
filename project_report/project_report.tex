\documentclass[12pt]{article}
\usepackage[margin=1in]{geometry}
\usepackage{amsmath, amssymb, amsthm}
\usepackage{tcolorbox}
\usepackage{lastpage}
\usepackage{fancyhdr, accents}
\usepackage{natbib}
\usepackage{graphicx}
\usepackage{hyperref}

\pagestyle{fancy}
\setlength{\headheight}{40pt}

\newcommand{\ubar}[1]{\underaccent{\bar}{#1}}

\newcommand\tab[1][1cm]{\hspace*{#1}}

\title{CSC111H1-S Winter 2021 - Fundamentals of Computer Science 2 \\ Course Project Report}
\author{
  Ching Chang\\
  \and
  Letian Cheng\\
  \and
  Arkaprava Choudhury\\
  \and
  Hanrui Fan
}
\date{\today}

\begin{document}

\maketitle

\newpage

\lhead{Ching Chang, Letian Cheng \\ Arkaprava Choudhury, Hanrui Fan}
\rhead{CSC111H1-S Winter 2021 \\ Fundamentals of Computer Science 2 \\ Course Project Report}
\cfoot{\thepage\ of \pageref{LastPage}}

\begin{enumerate}
\item \section*{Part 1}
\textbf{Animmend - An interactive anime recommendation system.}

Ching Chang, Letian Cheng, Arkaprava Choudhury, Hanrui Fan

\newpage

\item \section*{Part 2: Problem Description and Research Question}
\begin{enumerate}
    \item \textbf{Overview of Background}
    
    \quad Anime is a genre of film and television animation that originated in Japan. Its wide range of storyline and unique art style has attracted over 20 millions audiences since 2018 \citep{Ani18}, and has been significantly scaling in multiple directions in the past 17 years \citep{Ell18}. With over 210 anime published last year in 2020 \citep{wiki21}, there are currently over 17,548 anime in the world \citep{MyA21} — way too many for anyone to pick!
    
    \item \textbf{Problem and Motivation}
    
    \quad This is a problem for both the audience and the producers of the anime who put in their blood and sweat into creating the work, only for it to be buried down in the monstrous pile of anime that are endlessly being published. Although there are anime-recommendation websites and applications out there that recommend users anime they might like based on their watch history, they tend to favour the trend and popular choices. This causes the established studios to self-perpetuate on the top of the pyramid scheme, while the rest that fails to make a good first impression end up at the bottom of the iceberg, forgotten. Much like every other film and television animation, anime is a form of fine arts medium where the authors, artists, and the production team splash their canvas with imagination and creativity. In a field with such freedom and originality, every anime has a meaningful story for someone in the world. Even the most terribly rated anime by the general could be a treasure to the right audience. However, it is difficult finding this so-called “right audience”. Anime with big names speak for themselves, so what the anime community lacks is a bridge between the unheard anime and the audience.
    
    \item \textbf{Our Goal}
    
    \quad To help mitigate this problem, we seek to create \textbf{an application that recommends unpopular anime based on the user’s input anime and relevant themes that they might be interested in.}
\end{enumerate}

\newpage

\item \section*{Part 3: Datasets}

\begin{itemize}
    \item ...
\end{itemize}
\newpage

\item \section*{Part 4: Computational Overview}

\begin{text}

\begin{itemize}
    \item PROCESSING INPUT DATA
    
    ...
    \item GRAPH IMPLEMENTATION OVERVIEW
    
    ...
    \item INTERACTIVE PART
    
    ...
\end{itemize}

\end{text}

\newpage

\item \section*{Part 5: Instructions for obtaining datasets and running the program}

\begin{text}

...

\end{text}



\newpage

\item \section*{Part 6: Changes from Phase 1}

\begin{text}

...

\end{text}

\newpage

\item \section*{Part 7: Discussion}

\begin{text}

% discuss, analyse, and interpret the results of your program
%Do the results of your computational exploration help answer research question?
%What limitations did you encounter, with the datasets you found, the algorithms/libraries you used, or other obstacles?
%What are some next steps for further exploration?

\emph{\textbf{Do the results of your computational exploration help answer this question?}}

...

\emph{\textbf{What limitations did you encounter, with the datasets you found, the algorithms and libraries you used, or other obstacles?}}

...

\emph{\textbf{What are some next steps for further exploration?}}

...

\end{text}


\maketitle

\newpage

\bibliography{references}
\bibliographystyle{apalike}

\end{enumerate}

\end{document}